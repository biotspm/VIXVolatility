\documentclass[aspectratio=169]{beamer}

\usepackage{mystyle}

\title{Measuring the Information Content of VIX Volatility}
\author{Context: Humboldt Project\\
Supervisor: Prof. Franziska Peter\\
Author: Sophia Gläser (7. Semester BA CME)}
\date{\small \today}

\addbibresource{./bib/bibliography.bib} 

\begin{document}

\begin{frame}
\maketitle
\end{frame}

\begin{frame}
\frametitle{Table of Contents}
\tableofcontents
\end{frame}

\section{Introduction}

\begin{frame}
\frametitle{Motivation: Why this project? Why does Volatility matter?}
	\begin{itemize}
	\item<1-> Volatility is not the same as risk, but a closely related concept
	\begin{itemize}
	\item<2-> crucial input to risk measures, such as VaR
	\item<2-> used for pricing of financial instruments, such as derivatives
	\item<2-> used for risk-return trade-off and therefore management decisions
	\end{itemize}
	\end{itemize}
\end{frame}

\begin{frame}
\frametitle{More closely: What exactly is Volatility?}
	\begin{itemize}
	\item<1-> Volatility is usually understood as the standard deviation from the expected value 
	\item<1-> What causes asset price movement and thus volatility?
	\begin{itemize}
	\item<2-> as Eugene Fama said himself: We do not yet know in detail, how stock marked prices form 
	\item<2-> Assuming Market efficiency (\citeauthor{fama1970}): Stock prices incorporate available information from the market, because of competition and free entry 
	\item<2-> $\rightarrow$ Stock prices react to the market 
	\end{itemize}
	\end{itemize}
	
	\begin{itemize}
	\item<3-> What is ''normal`` volatility
	\begin{itemize}
	\item<4-> annual volatility.. monthly.. daily..
	\end{itemize}
	\end{itemize}
\end{frame}

\begin{frame}
\frametitle{The Problem: Why is it so hard to measure and forecast volatility?}
	\begin{itemize}
	\item<1-> volatility is not directly observable
	\begin{itemize}
	\item<1-> we can estimate it for a given time period, but every period has different information content (intraday, overnight)
	\end{itemize}
	\item<2-> Joint hypothesis problem
	\begin{itemize}
	\item<2-> Market efficiency per se is not testable
	\end{itemize}
	\end{itemize}
\end{frame}

\section{Data}

\begin{frame}
\frametitle{Volatility of S \& P 500}
	\begin{itemize}
	\item sample period: ..
	\item 
	\end{itemize}
\end{frame}


\section{Method}

\begin{frame}
\frametitle{}
	\begin{itemize}
	\item Regression of realized volatility on historic volatility
	\end{itemize}
\end{frame}


\section{Results so far}

\begin{frame}
\frametitle{}
	\begin{itemize}
	\item
	\end{itemize}
\end{frame}


\section{Possible Problems coming up}

\begin{frame}
\frametitle{Questions currently to solve}
	\begin{itemize}
	\item Having gathered all this information about volatility measurement, what is the most accurate way to set up my regression?
	
	\end{itemize}
\end{frame}

\section*{Sources}
\begin{frame}
\printbibliography
\end{frame}



\end{document}
