%!TEX root = ../Main.tex

\section{Results}\label{sec:5Results}

This section presents the estimation results. The first subsection reports the results obtained with the regression analysis described in section \ref{sec:42Method}. The second subsection provides the results of the robustness checks conducted. 

\subsection{Regression Analysis Results}\label{sec:51Regression}
In the following, the regression results are presented. The regression outputs can be found in table \ref{tab:newey2} for the logarithmic regression and \ref{tab:newey1} for the level regression, in the appendix. To make it easier to overview the results, only one regression table was included in this section. The logarithmic table is displayed, since as discussed in \ref{sec:41Data} it should be better specified, which is confirmed by it having a lower AIC than the level specification.\\
Firstly, if the VIX contains information about future realized volatility, the slope of the VIX in \ref{eq:Reg1b} and \ref{eq:Reg2b} should be positive and significantly different from zero. The $H_{0}: \beta^{VIX} = 0$ can be rejected for both the level and the logarithmic regression specification, which implies that the VIX contains significant information for future volatility. Thus the \textbf{\ac{H1}} can be confirmed.\\
Secondly, if the VIX has more explanatory value than the historic volatilities, the adjusted $R^{2}$ in the second regression with the \ac{VIX} (\ref{eq:Reg2a} and \ref{eq:Reg2b}) should be larger than in the first regression with the historic volatilities (\ref{eq:Reg1a} and \ref{eq:Reg1b}). This not true for both the level and the logarithmic specification. Even though the $R^{2}$'s are close, they are slightly larger in the first regression ($0.726$) than in the second ($0.693$). Thus, the \textbf{\ac{H2}} can not be confirmed.\\
Thirdly, if the VIX adds explanatory value to the historic volatilities, the adjusted $R^{2}$ in the third regression (\ref{eq:Reg3a} and \ref{eq:Reg3b}) with the VIX included should be larger than in the first regression, containing only the historic volatilities. This is true for both specifications, in alignment with \textbf{\ac{H3}}.\\
Finally, if the VIX subsumes all information contained in the historic volatilities, the historic volatilities should not be significantly different from zero, contrary to the VIX. Thus the $H_{0}: \beta^{d} = \beta^{w} = \beta^{m} = 0 \ and \ \beta^{VIX} = 1$ should not be rejected. The results from the F-test can be found in the appendix in table \ref{tab:ftest1} and \ref{ftest2}. The results show, that the $H_{0}$ can be rejected at the $0.001$ significance level for both specifications. Thus, \textbf{\ac{H4}} can not be confirmed.

%

% Table created by stargazer v.5.2.2 by Marek Hlavac, Harvard University. E-mail: hlavac at fas.harvard.edu
% Date and time: Fr, Jan 11, 2019 - 15:59:32
\begin{table}[!htbp] \centering 
  \caption{logarithmic regression} 
  \label{} 
\begin{tabular}{@{\extracolsep{5pt}}lccc} 
\\[-1.8ex]\hline 
\hline \\[-1.8ex] 
 & \multicolumn{3}{c}{\textit{Dependent variable:}} \\ 
\cline{2-4} 
\\[-1.8ex] & \multicolumn{3}{c}{Realized Volatility} \\ 
\\[-1.8ex] & (1) & (2) & (3)\\ 
\hline \\[-1.8ex] 
 Intercept & $-$0.043$^{***}$ & $-$0.406$^{***}$ & $-$0.186$^{***}$ \\ 
  & (0.007) & (0.017) & (0.014) \\ 
  & & & \\ 
 $RV^{d}_{log}$ & 0.344$^{***}$ &  & 0.264$^{***}$ \\ 
  & (0.027) &  & (0.025) \\ 
  & & & \\ 
 $RV^{w}_{log}$ & 0.395$^{***}$ &  & 0.285$^{***}$ \\ 
  & (0.035) &  & (0.035) \\ 
  & & & \\ 
 $RV^{m}_{log}$ & 0.208$^{***}$ &  & 0.015 \\ 
  & (0.024) &  & (0.029) \\ 
  & & & \\ 
 crisis & 0.020$^{*}$ & $-$0.224$^{***}$ & $-$0.099$^{***}$ \\ 
  & (0.012) & (0.034) & (0.016) \\ 
  & & & \\ 
 $VIX_{log}$ &  & 1.472$^{***}$ & 0.644$^{***}$ \\ 
  &  & (0.048) & (0.045) \\ 
  & & & \\ 
\hline \\[-1.8ex] 
AIC & 1874.2 & 2372.2 & 1555.5 \\ 
Observations & 4,434 & 4,433 & 4,433 \\ 
R$^{2}$ & 0.726 & 0.693 & 0.745 \\ 
Adjusted R$^{2}$ & 0.726 & 0.693 & 0.745 \\ 
Residual Std. Error & 0.299 (df = 4429) & 0.316 (df = 4430) & 0.288 (df = 4427) \\ 
\hline 
\hline \\[-1.8ex] 
\textit{Note:}  & \multicolumn{3}{r}{$^{*}$p$<$0.1; $^{**}$p$<$0.05; $^{***}$p$<$0.01} \\ 
\end{tabular} 
\end{table} 
\label{tab:newey2}



\subsection{Robustness Checks}\label{sec51Robustness}

\subsubsection{Monthly non-overlapping samples}
Previous samples testing the information content of (model-free) implied volatility often used overlapping samples, meaning that the same option is used in several implied-volatility calculations. However, \textcite{christensen1998} showed, that the use of overlapping samples creates a telescopic overlap problem and thus standard statistical inferences are no longer valid.\\
Therefore the same regression analysis was conducted using non-overlapping samples. \textcite{jiang2003} use monthly non-overlapping samples, using the first Wednesday of every month, since they calculate the implied volatility over a horizon on one month. The VIX however is calculated slightly differently. It contains near- and next-term options options between 23 and 37 days to maturity (which is always a Friday), and every week the options roll over to new maturities. For example, taking the second Tuesday in October, the near-term option expires in 24 days, and the next-term option in 31 days. One day later, the option that expires now in 30 days is the near-term option, and another option expiring in 37 days is the next-term option. This next-term option will, one week later, roll over to a near-term option and, one more week later, drop out of the calculation. Thus, an option can be included in the calculation for up to two weeks. Therefore, the regression is conducted with daily volatilities, but only for one value out of two weeks. As in \textcite{jiang2003}, the values of Wednesday are used, for each second week. \\
The estimation results for the sample using non-overlapping data are summarized in \ref{tab:overlap1} and \ref{tab:overlap2}.




\newpage
















