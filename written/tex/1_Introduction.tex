%!TEX root = ../Main.tex

\section{Introduction: The Importance of Volatility Estimation}\label{sec:1Intro}

Financial market volatility is of high interest for the financial sector. Asset return volatility is for example key input to the pricing of financial instruments like derivatives, or to risk measures such as the Value at Risk. Moreover they give information on the risk-return trade-off, which is a central question in portfolio allocation and managerial decision making. \\
As however volatility is not directly observable, it has to be estimated. Seeing its importance, it is not astonishing that during the last years, considerable research has been devoted to the question, how volatility can be estimated and predicted. Two prominent approaches that have been commonly used are time series models, like the ARCH or stochastic volatility models, or implied volatility models, like the \ac{BS} implied volatility models. Whereas time series models rely on historic data, implied volatility models use option price data\footnote{There are, of course, various other methods for volatility estimation and forecasting, such as various nonparametric methods or neural networks based models. However, they shall not be discussed here, for an encompassing overview of volatility estimation and forecasting can be found in \textcite{jiang2003}.}.\\
Out of these two, there has been a growing interest in implied volatility during the the recent years. Since options are contracts giving the holder the right to buy or sell an underlying asset at a specified date in the future, they are said to have a ``forward-looking nature'', meaning that they are supposed to be highly related to the market's expectation about the future volatility of the underlying asset over the remaining life of the option. Therefore, if market agents are rational (meaning that the market uses all available information to form it's expectations about future price movements and volatiltiy) markets are efficient and the model pricing the option is specified correctly, the volatility implied from option prices should be an unbiased and efficient estimator of future realized volatility \parencite{bakanova2010}. Moreover, it should be the best possible forecast possible, given the current information \parencite{christensen2002}. \\
For some time, one popular approach for estimating implied volatility has been the \ac{BS} implied volatility. The \ac{BS} model is an option pricing model, using volatility as an input factor. By using observed option prices as the input and solving for volatility, it is possible to obtain a volatility measure that is widely believed to be ``informationally superior to the historic volatiltiy of the underlying asset'' \parencite[p.1305]{jiang2003}. Early studies found this \ac{BS} implied volatility to be a biased forecast of realized volatility, not containing significantly more information than historic volatility. More recent studies however rejected thesse findings and presented evidence that there is indeed additional information contained in option prices \parencite{jiang2003}. A reason for this discrepancy could be that early studies did not consider several data and methodological problems, such as long enough time series, a possible regime shift around the crash in 1987 and the use of non-overlapping samples \parencite{jiang2003}. \citeauthor{christensen1998} for example took this into account and found that implied volatility outperforms historic volatility. All in all, this new insights provided evidence against the inefficiency of implied volatility.\\
However, even though \ac{BS} implied volatility is found to be the overall more efficient forecast of realized volatility compared to historic volatility, the \ac{BS} implied volatility has some specification problems. Firstly, \ac{BS} implied volatility focuses on at-the-money options. The advantage is, that at-the-money options are the once most actively traded and thus the most liquid ones. However this focus fails to include information contained in other options. Moreover, volatility estimation with the \ac{BS} model includes the same assumptions as are made in the \ac{BS} model itself. Thus, tests based on the \ac{BS} equation are joined tests of market efficiency (as market efficiency has to be assumped to use option prices for volatility estimation, as mentioned above) and the \ac{BS} model, and therefore suffer from model misspecification errors \parencite{jiang2003}. \\
That is why during the last years, implied volatility indexes which are not based on a pricing assumption have gained popularity. One of these model-free implied volatility indexes is the \ac{VIX} from \ac{CBOE}.





