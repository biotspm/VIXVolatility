%!TEX root = ../Main.tex

\section{Introduction}\label{sec:1Intro}
Financial market volatility is of high interest for the financial sector. Asset return volatility is for example key input to the pricing of financial instruments like derivatives, or to risk measures such as the Value at Risk. Moreover, it gives information on the risk-return trade-off, which is a central question in portfolio allocation and managerial decision making. \\
However, as volatility is not directly observable it has to be estimated. Seeing its importance, it is not astonishing that considerable research has been devoted to the question, how volatility can be estimated and predicted. Whereas earlier approaches were mainly ARCH or stochastic volatility models, using historic volatility, there has recently been a growing interest in volatility implied from option price data \parencite{bakanova2010}\footnote{There are, of course, various other methods for volatility estimation and forecasting, such as nonparametric methods or neural networks based models. An encompassing overview of volatility estimation and forecasting can be found in \textcite{jiang2003}.}.\\
Since options are contracts giving the holder the right to buy or sell an underlying asset at a specified date in the future, they are said to have a ``forward-looking nature'', meaning that they are supposed to be highly related to the market's expectation about the future volatility of the underlying asset over the remaining life of the option. Therefore, if market agents are rational (meaning that the market uses all available information to form its expectations about future price movements and volatility) markets are efficient and the model pricing the option is specified correctly, the volatility implied from option prices should be an unbiased and efficient estimator of future realized volatility \parencite{bakanova2010}.\\
One popular approach for estimating implied volatility is the \ac{BS} implied volatility. The \ac{BS} model is an option pricing model, using volatility as an input factor. By using observed option prices as the input and solving for volatility, it is possible to obtain a volatility measure that is widely believed to be ``informationally superior to the historic volatility of the underlying asset'' \parencite[p.1305]{jiang2003}. Whereas early studies found it to be a biased forecast, the correction of several methodological problems (such as long enough time series, the inclusion of crisis periods and the use of non-overlapping samples) provided evidence, that \ac{BS} implied volatility contains significant information content and is a more efficient forecast of realized volatility than historic volatility \parencite{jiang2003}.\\
However, the \ac{BS} implied volatility has some specification problems. Firstly, it focuses on at-the-money options. At-the-money options are usually the once most actively traded and thus the most liquid ones, but this fails to include information contained in other options. Moreover, volatility estimation with the \ac{BS} model includes the same assumptions as are made in the \ac{BS} model itself, thus, tests based on the \ac{BS} equation are joined tests of market efficiency and the option pricing model, and therefore suffer from model misspecification errors \parencite{jiang2003}. \\
That is why during the last years, implied volatility indexes which are not based on a pricing assumption have gained popularity. The idea was introduced by \textcite{britten2000}, extending the work of \textcite{derman1994} \textcite{dupire1994}, \textcite{dupire1997} and \textcite{rubinstein1994} on implied distributions. Contrary to the previously described \ac{BS} implied volatility, their approach is derived directly from option prices and the no-arbitrage condition. Showing, that the risk-neutral return variance can be derived entirely from option price data, they provide a volatility measure which does not suffer from the joined hypothesis problem and moreover allows to incorporate not only at-the-money options.
Several papers used the results from \textcite{britten200}, extended their approach and tested the informational efficiency of the model-free implied volatility, such as \textcite{bakanova2010}, \textcite{taylor2010} or \textcite{jiang2003}. \textcite{jiang2003}, for example, find that it subsumes all information contained in both \ac{BS} and historic volatility using an OLS regression and variance data. However, in the OLS regression using volatility data and log specifications, the model-free implied volatility is not more significant than historic volatility.
In 2003, \ac{CBOE} reacted to this overall confiming academic findings and changed the calculation of its \ac{VIX} from the \ac{BS} to the model-free implied volatility. This paper will directly use the \ac{VIX} index from \ac{CBOE} and test the data for a 17-year time period, including the financial crisis around 2008, using both univariate and encompassing regression analysis. Moreover, the historic volatility information included in the model is extended. Previous research mainly used only the one-day-lagged realized volatility, whereas this paper uses the approach from the HAR-RV model from \textcite{corsi2009}, and includes equally weekly and monthly historic volatility averages in the regression. 





