%!TEX root = ../Main.tex

\section{Introduction: The Importance of Volatility Measurement}\label{sec:1Intro}

\subsection{Why Volatility matters: Volatility as Key Input to Option Pricing Models and Risk Measures}\label{sec:11WhyVola}
Distributional characteristics of asset returns are of high interest for the financial sector. They are for example key input to the pricing of financial instruments like derivatives, or to risk measures, such as the Value-at-Risk. Moreover they give information on the risk-return trade-off, which is a central question in portfolio allocation and managerial decision making. Of particular interest is the asset's return volatility, being the most dominant time-varying distribution characteristic. \parencite{andersen2003}.\\
Risk is important because.. . The concept of risk is closely linked to that of volatility, as the second moment characterstic of asset return distributions. Many pricing models measure risk through volatility, which thus influences the expected return \parencite{harvey1992}. Moreover many risk-measure, such as the Value-at-Risk are very closely related to volatility.  (Alternative)\\
Volatility ``seeks to capture the strength of the (unexpected) return variation over a given period of time'' \parencite[p.7]{andersen2001}\\
As volatility is not directly observable, it has to be estimated. During the last years, considerable research has been devoted to the question, how volatility can be measured or estimated. Two prominent categories of approaches are on the one hand time series models using historic volatility, and on the other hand implied models using option price data\footnote{There are, of course, various other methods, such as nonparametric methods or neural networks based models \parencite{jiang2003}, however they shall not be discussed here}. In the recent years, Black-and-Scholes implied volatility measurement gained popularity, this approach uses the forward-looking nature of option prices. Options are contracts, giving the holder the right to either buy (call option), or sell (put option), an underlying asset, at a specified date in the future for a certain price \parencite{hull2006}. Assuming rational agents/expectation, the market uses all available information to form it's expectation about future price movements and thus about volatility. Assuming furthermore that the market is efficient (meaning as Eugene Fama defined it, that prices reflect all available information), the market's estimate of future volatility is the best possible forecast possible, given the current information \parencite{christensen2002}. Due to this forward looking component of option contracts, option prices indirectly contain the market participants' expectations of the underlying asset's future movements. A widely used model to price this option contracts is the Black-and-Scholes-Merton model, which uses the option's volatility as an input factor. By using observed option prices as the input and solving for volatility, it is possible to obtain a volatility measure that is widely believed to be ``informationally superior to the historic volatiltiy of the underlying asset'' \parencite[p.1305]{jiang2003}.\\
Early studies found implied volatility to be a biased forecast of realized volatility, not containing significantly more information than historic volatility. More recent studies however presented evidence that there is important information contained in option prices, that adds to the efficiency of volatility forecasting when implied volatility is included \parencite{jiang2003}. A reason for this discrepancy in results could be that early studies did not consider several data and methodological problems, such as long enough time series, a possible regime shift around the crash in 1987 and the use of non-overlapping samples \parencite{jiang2003}. \citeauthor{christensen1998} for example took this into account and found that implied volatility outperforms historic volatility. All in all \citeauthor{jiang2003} summarize, that collectively ``these studies present evidence that implied volatility a more efficient forecast for future volatility than historic volatility'' (p.1306).

\subsection{Weaknesses of Existing Models: VIX Introduced by CBOE}\label{sec:12Weakness}
Even though BS implied volatility is found to be the overall more efficient forecast of realized volatility compared to historic volatility \parencite{jiang2003}, the \gls{BS} implied volatility has some specification problems. Firstly, \gls{BS} implied volatility focuses on at-the-money options. The advantage is, that at-the-money options are the once most actively traded and thus the most liquid ones. However this focus fails to include information contained in other options. Moreover, volatility estimation with the \gls{BS} model, includes the same assumptions as are made in the \gls{BS} model itself. Thus tests based on the \gls{BS} equation are actually joined tests of market efficiency (as market efficiency is assumped to use option prices for volatility estimation, as mentioned above) and the \gls{BS} model, and therefore suffer from a model misspecification error \parencite{jiang2003}. \\
That is why during the last years, implied volatility indexes which are not based on a pricing assumption gained popularity. One of these model-free implied volatility indexes is the VIX from CBOE.

\begin{itemize}\itemsep0pt
\item power of volatility models lies in out-of-sample forecasting power
\item so far \gls{BS} implied volatility models had the best out of sampling forecasting power, but they have several problems (most importantly joint hypothesis problem) 
\end{itemize}



