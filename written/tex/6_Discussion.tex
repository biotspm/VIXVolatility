%!TEX root = ../Main.tex

\section{Conclusion}\label{sec:6Discussion}
%QUICK SUMMARY%
This paper analyses the information content for one-day ahead realized volatility of the \ac{VIX}, as an example for model-free implied volatility, and compares it to historic volatilities, using daily, weekly and monthly historic volatility. After introducing different methods for volatility measurement and modelling, OLS regression is used to evaluate the information content. Apart from extending the regression approach from previous research, the paper follows the methodology corrections proposed to improve the statistical validity of the results, namely using long time series, high-frequency returns, non-overlapping samples and the inclusion of financial crisis periods.\\
%CONSISTENT AND "POSTIVE" FINDINGS%
The results show, that the \ac{VIX} contains significant information content for future realized volatility, also beyond the information included in the historic volatilities. These findings are consistent with previous research and robust to serial correlation and alternative sampling methods. \\
Moreover, this paper shows, that not only the daily, but also the weekly average of the  historic volatility contains significant information content for future volatility and it is useful to include them in the model. Comparing the models, the logarithmic specification always performed better than the level specification, which was to be expected seeing the descriptive statistics of the data. \\
%CONTRARY FINDINGS AND POSSIBLE REASONS%
However, it can not be concluded, that the model-free implied volatility subsumes all the information contained in the historic volatility, which partly contradicts previous research. The discrepancy could be due to several reasons. This paper used the approach from the HAR-RV model, but as daily volatility was significant in every regression that is not likely to be the reason. Alternatively, an explanation could be the different and longer time period considered, including and accounting for a financial crisis. Moreover, \textcite{jiang2003} could show only in the regression using the variances, that the model-free implied volatility subsumed the information contained in historic volatility. With volatilities, the model-free implied volatility was insignificant in the model containing all explanatory variables, and the $R^{2}$ did not increase either.\\
To conclude, the results show, that model-free implied volatility and the VIX contain significant information content for future realized volatility, but they do not fully subsume the information content from historic volatilities.\\
An interesting outlook for future research could be to further test the informational efficiency of model-free implied volatility using not index, but single stock data, as \textcite{taylor2010} did, and to further investigate the reason for the slight discrepancies found in current research.

%Contrary to the findings of \textcite{jiang2003} and \textcite{bakanova2010}, but in alignment with \textcite{taylor2010}
%However, Both \textcite{jiang2003} and \textcite{bakanova2010} based their results of the paper of \textcite{britten2000}, this paper used the \ac{VIX}, with slight differences. 