% A - E

@article{andersen2001,
author = {Andersen, Torben G. and Bollerslev, Tim and Diebold, Francis X},
file = {:C$\backslash$:/Users/Sophia/Dropbox/01{\_}Studium/02{\_}Semester/7.{\_}Semester{\_}Fall{\_}18/Humboldt/Literatur/Andersen et. al{\_} PARAMETRIC AND NONPARAMETRIC VOLATILITY MEASUREMENT.pdf:pdf},
journal = {Handbook of Financial Econometrics},
number = {March},
title = {{Parametric and Nonparametric Volatility Measurement}},
year = {2001}
}

@article{andersen2008,
abstract = {The notion of model-free implied volatility (MFIV), constituting the basis for the highly publicized VIX volatility index, can be hard to measure with accuracy due to the lack of precise prices for options with strikes in the tails of the return distribution. This is reflected in practice as the VIX index is computed through a tail-truncation which renders it more compatible with the related concept of corridor implied volatility (CIV). We provide a comprehensive derivation of the CIV measure and relate it to MFIV under general assumptions. In addition, we price the various volatility contracts, and hence estimate the corresponding volatility measures, under the standard Black-Scholes model. Finally, we undertake the first empirical exploration of the CIV measures in the literature. Our results indicate that the measure can help us refine and systematize the information embedded in the derivatives markets. As such, the CIV measure may serve as a tool to facilitate empirical analysis of both volatility forecasting and volatility risk pricing across distinct future states of the world for diverse asset categories and time horizons.},
author = {Andersen, Torben G. and Bondarenko, Oleg},
doi = {10.2139/ssrn.1150136},
file = {:C$\backslash$:/Users/Sophia/Dropbox/01{\_}Studium/02{\_}Semester/7.{\_}Semester{\_}Fall{\_}18/Humboldt/Literatur/Andersen et al.{\_}CONSTRUCTION AND INTERPRETATION OF MODEL-FREE IMPLIED VOLATILITY.pdf:pdf},
isbn = {w13449},
issn = {1556-5068},
journal = {Ssrn},
keywords = {C53,Corridor Implied Volatility,G12,G13,Model-Free Implied Volatility,Realized Volatility,Risk-Neutral Density,VIX,Volatility Forecasting},
number = {August},
title = {{Construction and Interpretation of Model-Free Implied Volatility}},
year = {2008}
}

@article{andersen2003,
author = {Andersen, Torben G and Bollerslev, Tim and Diebold, Francis X and Labys, Paul and Diebold, Francis X and Labys, Paul},
file = {:C$\backslash$:/Users/Sophia/Dropbox/01{\_}Studium/02{\_}Semester/7.{\_}Semester{\_}Fall{\_}18/Humboldt/Literatur/Andersen et.al{\_}Modeling and Forecasting Realized volatility.pdf:pdf},
number = {2},
pages = {579--625},
title = {{Modeling and Forecasting Realized Volatility Published by : The Econometric Society Stable URL : http://www.jstor.org/stable/3082068 The Econometric Society is collaborating with JSTOR to digitize , preserve and extend access to Econometrica}},
volume = {71},
year = {2018}
}

@article{bakanova2010,
abstract = {The information content of implied volatility in the crude oil market},
author = {Bakanova, Asyl},
file = {::},
journal = {Proceedings of the Brunel conference},
keywords = {crude oil,extreme value,implied volatility,volatility forecasts},
pages = {1--19},
title = {{The information content of implied volatility in the crude oil market}},
url = {http://qass.org.uk/2010-May{\_}Brunel-conference/Bakanova.pdf},
year = {2010},
organization={Citeseer}
}


@article{behrendt2018,
author = {Behrendt, Simon and Schmidt, Alexander},
doi = {10.1016/j.jbankfin.2018.09.016},
file = {:C$\backslash$:/Users/Sophia/Dropbox/01{\_}Studium/02{\_}Semester/7.{\_}Semester{\_}Fall{\_}18/Humboldt/Literatur/1-s2.0-S0378426618302115-main.pdf:pdf},
issn = {03784266},
journal = {Journal of Banking {\&} Finance},
keywords = {Return volatility,Investor sentiment,Twitter,Intra},
pages = {355--367},
publisher = {Elsevier B.V.},
title = {{The Twitter Myth Revisited: Intraday Investor Sentiment, Twitter Activity and Individual-Level Stock Return Volatility}},
url = {https://linkinghub.elsevier.com/retrieve/pii/S0378426618302115},
volume = {96},
year = {2018}
}

@article{black1973,
  title={The pricing of options and corporate liabilities},
  author={Black, Fischer and Scholes, Myron},
  journal={Journal of political economy},
  volume={81},
  number={3},
  pages={637--654},
  year={1973},
  publisher={The University of Chicago Press}
}

@article{britten2000,
abstract = {ABSTRACT $\backslash$nThis paper characterizes all continuous price processes that are consistent with $\backslash$ncurrent option prices. This extends Derman and Kani (1994), Dupire (1994, 1997), $\backslash$nand Rubinstein (1994), who only consider processes with deterministic volatility. $\backslash$nOur characterization implies a volatility forecast that does not require a specific $\backslash$nmodel, only current option prices. We show how arbitrary volatility processes can $\backslash$nbe adjusted to fit current option prices exactly, just as interest rate processes can $\backslash$nbe adjusted to fit bond prices exactly. The procedure works with many volatility $\backslash$nmodels, is fast to calibrate, and can price exotic options efficiently using familiar $\backslash$nlattice techniques.},
author = {Britten-Jones, M. and Neuberger, a.},
file = {:C$\backslash$:/Users/Sophia/Dropbox/01{\_}Studium/02{\_}Semester/7.{\_}Semester{\_}Fall{\_}18/Humboldt/Literatur/0022-1082.00228.pdf:pdf},
journal = {The Journal of Finance},
number = {2},
pages = {839--866},
title = {{Option prices, implied processes, and stochastic volatility}},
volume = {55},
year = {2000},
publisher={Wiley Online Library}
}

@article{canina1993,
  title={The informational content of implied volatility},
  author={Canina, Linda and Figlewski, Stephen},
  journal={The Review of Financial Studies},
  volume={6},
  number={3},
  pages={659--681},
  year={1993},
  publisher={Oxford University Press}
}

@article{christensen1998,
  title={The relation between implied and realized volatility1},
  author={Christensen, Bent J and Prabhala, Nagpurnanand R},
  journal={Journal of financial economics},
  volume={50},
  number={2},
  pages={125--150},
  year={1998},
  publisher={Elsevier}
}

@article{christensen2002,
  title={New evidence on the implied-realized volatility relation},
  author={Christensen, Bent Jesper and Hansen, Charlotte Strunk},
  journal={The European Journal of Finance},
  volume={8},
  number={2},
  pages={187--205},
  year={2002},
  publisher={Taylor \& Francis}
}

@article{corsi2009,
abstract = {The paper proposes an additive cascade model of volatility components defined over different time periods. This volatility cascade leads to a simple AR-type model in the realized volatility with the feature of considering different volatility components realized over different time horizons and thus termed Heterogeneous Autoregressive model of Realized Volatility (HAR-RV). In spite of the simplicity of its structure and the absence of true long-memory properties , simulation results show that the HAR-RV model successfully achieves the purpose of reproducing the main empirical features of financial returns (long memory, fat tails, and self-similarity) in a very tractable and parsimonious way. Moreover, empirical results show remarkably good forecasting performance.},
author = {Corsi, Fulvio},
doi = {10.1093/jjfinec/nbp001},
file = {:C$\backslash$:/Users/Sophia/AppData/Local/Mendeley Ltd./Mendeley Desktop/Downloaded/Corsi - 2009 - A Simple Approximate Long-Memory Model of Realized Volatility.pdf:pdf},
journal = {Journal of Financial Econometrics},
number = {2},
pages = {174--196},
title = {{A Simple Approximate Long-Memory Model of Realized Volatility}},
volume = {7},
year = {2009}
}

@article{day1992,
abstract = {Previous studies of the information content of the implied volatilities from the prices of call options have used a cross-sectional regression approach. This paper compares the information content of the implied volatilities from call options on the S{\&}P 100 index to GARCH (Generalized Autoregressive Conditional Heteroscedasticity) and Exponential GARCH models of conditional volatility. By adding the implied volatility to GARCH and EGARCH models as an exogenous variable, the within-sample incremental information content of implied volatilities can be examined using a likelihood ratio test of several nested models for conditional volatility. The out-of-sample predictive content of these models is also examined by regressing ex post volatility on the implied volatilities and the forecasts from GARCH and EGARCH models. {\textcopyright} 1992.},
author = {Day, Theodore E. and Lewis, Craig M.},
doi = {10.1016/0304-4076(92)90073-Z},
file = {:C$\backslash$:/Users/Sophia/Dropbox/01{\_}Studium/02{\_}Semester/7.{\_}Semester{\_}Fall{\_}18/Humboldt/Literatur/Day Louis - Stock market volatility and the information content of stock index options.pdf:pdf},
isbn = {0304-4076},
issn = {03044076},
journal = {Journal of Econometrics},
number = {1-2},
pages = {267--287},
title = {{Stock market volatility and the information content of stock index options}},
volume = {52},
year = {1992}
}

@article{derman1994,
  title={Riding on a smile},
  author={Derman, Emanuel and Kani, Iraj},
  journal={Risk},
  volume={7},
  number={2},
  pages={32--39},
  year={1994}
}

@article{diMatteo2007,
abstract = {The most suitable paradigms and tools for investigating the scaling structure of financial time series are reviewed and discussed in the light of some recent empirical results. Different types of scaling are distinguished and several definitions of scaling exponents, scaling and multi-scaling processes are given. Methods to estimate such exponents from empirical financial data are reviewed. A detailed description of the Generalized Hurst exponent approach is presented and substantiated with an empirical analysis across different markets and assets.},
author = {{Di Matteo}, T.},
doi = {10.1080/14697680600969727},
file = {:C$\backslash$:/Users/Sophia/Dropbox/01{\_}Studium/02{\_}Semester/7.{\_}Semester{\_}Fall{\_}18/Humboldt/Literatur/RQUF{\_}A{\_}196874{\_}O.pdf:pdf},
isbn = {1469-7688},
issn = {14697688},
journal = {Quantitative Finance},
keywords = {Econophysics,Multifractal formalisms,Scaling,Time series analysis},
number = {1},
pages = {21--36},
title = {{Multi-scaling in finance}},
volume = {7},
year = {2007}
}

@article{dumas1998,
  title={Implied volatility functions: Empirical tests},
  author={Dumas, Bernard and Fleming, Jeff and Whaley, Robert E},
  journal={The Journal of Finance},
  volume={53},
  number={6},
  pages={2059--2106},
  year={1998},
  publisher={Wiley Online Library}
}

@article{dupire1994,
  title={Pricing with a smile},
  author={Dupire, Bruno and others},
  journal={Risk},
  volume={7},
  number={1},
  pages={18--20},
  year={1994}
}

@article{dupire1997,
  title={Pricing and hedging with smiles},
  author={Dupire, Bruno},
  journal={Mathematics of derivative securities},
  volume={1},
  number={1},
  pages={103--111},
  year={1997},
  publisher={Cambridge University Press Cambridge, UK}
}

@article{engle2012a,
abstract = {This paper proposes a new intraday volatility forecasting model, particularly suitable for modeling a large number of assets. We decompose volatility of high-frequency returns into components that may be easily interpreted and estimated. The conditional variance is a product of daily, diurnal, and stochastic intraday components. This model is applied to a comprehensive sample consisting of 10-minute returns on more than 2500 US equities. Apart from building a new model, we obtain several interesting forecasting results. We apply a number of different specifications. We estimate models for separate companies, pool data into industries, and consider other criteria for grouping returns. In general, forecasts from pooled cross-section of companies outperform the corresponding forecasts from company-by-company estimation. For less liquid stocks, however, we obtain better forecasts when we group less frequently traded companies together. (JEL: C22, C51, C53, G15)},
author = {Engle, Robert F and Sokalska, Magdalena E},
doi = {10.1093/jjfinec/nbr005},
journal = {JournalofFinancialEconom etrics},
number = {1},
pages = {54--83},
title = {{Forecasting intraday volatility in the US equity market. Multiplicative component GARCH Downloaded from}},
volume = {10},
year = {2012}
}

@article{exchange2009,
  title={The CBOE volatility index-VIX},
  author={Exchange, Chicago Board Options},
  journal={White Paper},
  pages={1--23},
  year={2009}
}

% F - J

@article{furfine1999,
abstract = {This paper examines the likelihood that failure of one bank would cause the subsequent collapse of a large number of other banks.  Using unique data on interbank payment flows, the magnitude of bilateral federal funds exposures is quantified.  These exposures are used to simulate the impact of various failure scenarios, and the risk of contagion is found to be economically small.},
author = {Furfine, Craig},
doi = {10.2139/ssrn.169089},
file = {:C$\backslash$:/Users/Sophia/Dropbox/01{\_}Studium/02{\_}Semester/7.{\_}Semester{\_}Fall{\_}18/Humboldt/Literatur/Furfine{\_}Interbank Exposures - Quantifying the Risk of Contagion.pdf:pdf},
isbn = {0022-2879 ER},
issn = {1538-4616},
journal = {Ssrn},
keywords = {G21},
number = {1},
pages = {111--128},
title = {{Interbank Exposures: Quantifying the Risk of Contagion}},
volume = {35},
year = {1999}
}

@article{harvey1992,
  title={Market volatility prediction and the efficiency of the S\&P 100 index option market},
  author={Harvey, Campbell R and Whaley, Robert E},
  journal={Journal of Financial Economics},
  volume={31},
  number={1},
  pages={43--73},
  year={1992}
}

@techreport{heber2009,
autor = {Heber, Gerd, Asger Lunde, Neil Shephard and Kevin Sheppard},
year = {2009},
title = {Oxford-Man Institute's realized library},
publisher = {Oxford-Man Institute, University of Oxford},
howpublished = "\url{https://realized.oxford-man.ox.ac.uk/data}",
volume = {Library Version 0.3},
note = "[Online; accessed 31-October-2018]"
}

@misc{hull2006,
  title={Options, futures, and other derivatives},
  author={John, C and others},
  year={2006},
  publisher={Pearson/Prentice Hall}
}

@article{jiang2005,
author = {{Jiang, George J {\&} Tian}, Yisong S.},
doi = {10.1093/rfs/hhi027},
file = {:C$\backslash$:/Users/Sophia/Dropbox/01{\_}Studium/02{\_}Semester/7.{\_}Semester{\_}Fall{\_}18/Humboldt/Literatur/Jiang, Tian{\_}Model Free Implied Volatility and its information content.pdf:pdf},
isbn = {5206213373},
issn = {0893-9454},
journal = {The Review of Financial Studies},
number = {4},
title = {{Model-Free Implied Volatility and Its Information Content}},
volume = {18},
year = {2003},
publisher={Oxford University Press}
}

% K  - O

@article{kambouroudis2015,
abstract = {This paper investigates the information content of implied volatility forecasts in the context of forecasting stock index return volatility by studying a number of US and European indices. Using a number of different autoregressive models for forecasting implied volatility, we examine whether implied volatility forecasts contain any additional information useful to predict future volatility beyond that embedded in GARCH models and realized volatility. The results show that implied volatility follows a predictable pattern and confirms previous literature that there is a contemporaneous relationship between implied volatility and index returns. When the predictive power of the implied volatility for future volatility is assessed, it is found that, overall, implied volatility contains additional information compared to GARCH and realized volatility, although individually performs worse. Nevertheless, a model that combines the information contained in an asymmetric GARCH model with the information from option markets and realized volatility through (asymmetric) ARMA models is the most appropriate for predicting future volatility. This evidence is also supported in the context of value-at-risk.},
annote = {Looks interesting and might be possible to replicate - read this next!!! 
},
author = {Kambouroudis, Dimos and Mcmillan, David and Tsakou, Katerina},
doi = {10.1002/fut.21783},
file = {::},
title = {{Forecasting stock return volatility: a comparison of GARCH, implied volatility and realized volatility models}},
year = {2015}
}


@article{mueller1993,
  title={Fractals and intrinsic time: A challenge to econometricians},
  author={M{\"u}ller, Ulrich A and Dacorogna, Michel M and Dav{\'e}, Rakhal D and Pictet, Olivier V and Olsen, Richard B and Ward, J Robert},
  journal={Unpublished manuscript, Olsen \& Associates, Z{\"u}rich},
  year={1993}
}

% P - Z

@article{poon2003,
author = {Poon, Ser-huang and Granger, Clive W.J.},
file = {:C$\backslash$:/Users/Sophia/Dropbox/01{\_}Studium/02{\_}Semester/7.{\_}Semester{\_}Fall{\_}18/Humboldt/Literatur/PoonGranger(2003).pdf:pdf},
journal = {Journal of Economic Literature},
number = {June},
pages = {478--539},
title = {{Forecasting volatility in financial markets : A review}},
volume = {41},
year = {2003}
}

@article{rubinstein1994,
  title={Implied binomial trees},
  author={Rubinstein, Mark},
  journal={The Journal of Finance},
  volume={49},
  number={3},
  pages={771--818},
  year={1994},
  publisher={Wiley Online Library}
}

@book{tsay2005,
  title={Analysis of financial time series},
  author={Tsay, Ruey S},
  volume={543},
  year={2005},
  publisher={John Wiley \& Sons}
}

@article{whaley1993,
  title={Derivatives on market volatility: Hedging tools long overdue},
  author={Whaley, Robert E},
  journal={The journal of Derivatives},
  volume={1},
  number={1},
  pages={71--84},
  year={1993},
  publisher={Institutional Investor Journals Umbrella}
}

@techreport{whaley2008,
abstract = {In the recent weeks of market turmoil, financial news services have begun routinely reporting the level of the CBOE's Market Volatility Index or "VIX", for short. While this new practice is healthy in the sense that investors are asking for more information in helping to assess the state of the current economic environment and to guide through turbulent waters, it is important to understand exactly what the index means in order to fully capture its usefulness to the market and to avoid misunderstanding and misconception. The purpose of paper is to describe the VIX and its history and purpose, and to explain how it fits within the array of indexes that help describe where the economy stands relative to other points in recent decades.},
author = {Whaley, Robert E},
file = {::},
title = {{Understanding VIX}},
year = {2008}
}



